\documentclass{report}

\usepackage[margin=1in]{geometry}
\usepackage{amsmath,amssymb}
%\usepackage{dsfont} %install texlive-fonts-extra 
\usepackage{tikz}
\usetikzlibrary{bayesnet}
\usepackage{setspace}
\usepackage{xcolor}

\author{Otto Fabius}
\title{SGVB Topic Modeling}
\newcommand{\specialcell}[2][c]{%
	\begin{tabular}[#1]{@{}l@{}}#2\end{tabular}}
\begin{document}
\large
\doublespacing
\maketitle

\begin{abstract}
	content...
\end{abstract}
\chapter*{}
\onehalfspacing
\section*{List of Symbols}
We use lowercase bold-faced symbols to denote vectors, and uppercase bold-faced symbols to denote matrices. \\ \\
\begin{tabular}{r l}
	\hspace{15mm} $\mathbf{x}$ & Data point \\
	$\mathbf{\hat{x}}$ & Data point normalized to have unit length \\	
	$\theta$ &  Model parameters \\
	$\phi$ & Parameters of approximate posterior model \\
	$z$ & Stochastic latent variable\\
	$\mu$ & Mean of stochastic latent variable\\
	$\sigma ^2 $ & Variance of stochastic latent variable \\
	$\mathcal{L}$ & Lower Bound \\
	$\tilde{\mathcal{L}}$ & Lower Bound estimate\\
	$\tilde{\mathcal{L}_w}$ & Per-word lower bound estimate \\
	$D_{KL}$ & Kullback-Leibler Divergence \\
	$V$ & Number of unique words in a dataset. \\
	$N_d$ & Number of documents in a dataset. \\
	$F$ & Matrix of node feature vectors
\end{tabular}
\\ \\
 We use index $i$ to indicate a data point, index $k$ to indicate a feature, and index $j$ to indicate a latent variable. In all applications in this thesis, our data points are documents represented by count vectors. Therefore, for example, $\mathbf{\hat{x}}_{ik}$ is the relative frequency of word $k$ in document $i$.
\section*{List of Abbreviations}
\begin{tabular}{r l}
	\hspace{10mm} BoW & Bag-of-Words \\
	DEF & Deep Exponential Families \\
	GCE & Graph Convolutional Encoder \\
	LDA & Latent Dirichlet Allocation \\
	SGVB & Stochastic Gradient Variational Bayes \\
	VAE & Variational Autoencoder \\
\end{tabular}

\tableofcontents

\doublespacing
\chapter{Introduction}
Aim, scope and structure of the thesis. 
\section{Research question}
Explain that we want to use sgvb for topic modelling. Introduce research question:
Can we perform large-scale, efficient, high-quality inference on bag-of-words representations of documents with sgvb? \\
Briefly discuss the advantages of this approach compared to other methods in topic modelling (one paragraph)
-	How do we deal with large vocabulary size?\\
-	What consequences do sparsity have/how can they be overcome? \\
-	What is learned in the (continuous) latent representation?



\pagebreak 
\nocite{*}
\bibliographystyle{amsplain}

\chapter{Background}
\section{Variational Inference}
\begin{itemize}
	\item Variational Optimization key idea
	\item Often used for inference problems, decompose the log marginal according to Bishop (eq 10.2)
	\item 
\end{itemize}
Variational Inference methods in general are used when data $\mathbf{X}$ is assumed to be generated through some process from an underlying set of stochastic latent variables $\mathbf{Z}$. The probability of data is therefore modeled as $p(\mathbf{X}) = p(\mathbf{Z}|\mathbf{X})p(\mathbf{Z})$. Moreover, this set of methods is used when the true posterior $p(Z|X)$ can not be evaluated analytically. Variational Bayes introduces a tractable distribution $q(\mathbf{\mathbf{Z}|\mathbf{X}})$ as an approximation to $p(\mathbf{Z}|\mathbf{X}))$.
\\
More on Variational Inference....
\section{Bag-of-words Topic Modelling}

\subsection{LSI and pLSI}
\subsection{LDA}
Explain LDA following (p)LSI and Variational Inference.
\subsection{Deep Exponential Families}

Deep exponential families is relevant for this work for two reasons: It shows that deeper models can be more powerful than LDA, and is currently the state of the art. Nothing in our methods depends on this section.


\section{SGVB}\label{sgvb_section}

Discuss requirements of problem scenario for SGVB to be applicable. \\

Notes:\\
- No simplifying assumptions are made about the marginal or posterior probabilities, as is the case in other VB methods (check!!) \\
- $q(\mathbf{Z}|\mathbf{X})$ is not necessarily factorial and its parameters $\phi$ are not computed from some closed-form expectation (as in mean field VI)
- General purpose introduction of sgvb . \\

- areas of success of sgvb.
\subsection{Neural Variational Inference for Topic Models}

\section{Stick-Breaking VAE}\label{sbvae_section}
	As detailed in \ref{sgvb_section}, one restriction of SGVB in general is the need for a differentiable, non-centered parametrization of the latent variables. Therefore, e.g. Beta distributed latent variables (as used in e.g. LDA \cite{bleil2003latent}), can not be used in the SGVB framework. \\ Nalisnyck \& Smith extend SGVB to Stick-breaking priors by using the Kumaraswami Distribution as the approximate posterior. The rest of this section \ref{sbvae_section} is not much more than a description of the relevant part of Nalisnyck \& Smyth \cite{nalisnick2016deep}.
	
	
	\subsection{The Stick-Breaking Process}\label{sb_process}
	
	\subsection{The Kumaraswami Distribution}\label{kum}
	

\section{Graph Convolutional Networks}\label{GCN_section}
Keep it brief and explain only the relevant stuff of Kipf and Welling for the section in models. 
\chapter{SGVB Topic Models}
In this Chapter, we present in detail the models and methods we use in our experiments. 

\section{Topic VAE}\label{TopicVAE}

In this section we describe our initial approach for topic modeling with SGVB. It does not deviate conceptually from the VAE model used in the experiments by Kingma and Welling\cite{kingma2013auto}, so one might call this a Topic VAE. We will describe the application-specific choices made within the general SGVB framework as described in section \ref{sgvb_section}, and derive the objective function used for optimization. 

\subsection{Model Specifics}

Within the general SGVB framework described in the previous chapter, we specify the following characteristics of our VAE Topic model:

\begin{enumerate}
	\item The representation of our documents $\mathbf{X}$
	\item The encoder $q(\mathbf{z}|\mathbf{x})$
	\item $p(z)$, a prior  over the latent variables
	\item A differentiable way of sampling $\mathbf{z}$ from $p(\mathbf{z}|\mathbf{x})$. 
	\item The decoder $p(\mathbf{x}|\mathbf{z})$
\end{enumerate}

The representation of the documents is a normalized Bag-of-Words representation s.t. document $i$ is represented by a unit vector $\hat{\mathbf{x}_i} = \frac{\mathbf{x}_i}{\sum_{k=1}^{V}x_{ik}}$. Although normalizing data is standard practice in neural network approaches (references...), typically each feature is normalized separately. In our approach, however, all features (word counts) of a data point (document) are normalized s.t. they represent word probabilities. This representation no longer contains information on the length of documents, which arguably weakly relates to topics.
\\
The encoder $q(\mathbf{z}|\mathbf{x})$ is a fully connected neural network with one or more hidden layers with ReLu activation functions. The input is $\mathbf{d}$ and the output is the mean and log standard deviation \{$\boldsymbol{\mu}, \log \boldsymbol{\sigma} ^2\}$ of the Multivariate Gaussian $N(\boldsymbol{\mu}, \boldsymbol{\sigma} ^2\textbf{I})$. For one hidden layer this would be the following function:
\begin{align}
\mathbf{h_{e1}} = \text{ReLu}(\mathbf{\hat{x}}\mathbf{W_{e1}} + \mathbf{b}) \label{he1}\\
\boldsymbol{\mu} = \mathbf{h_{e1}W}_{\mu} \label{vae_encoding_mu} \\
\log \boldsymbol{\sigma}^2 = \mathbf{h_{e1}W}_{\sigma} \label{vae_encoding_sig}
\end{align} 
We use prior $p(\mathbf{z}) = N(0,\textbf{I})$
\\
Our encoder and prior are identical to the experiments in \cite{kingma2013auto}, and so is our (differentiable) sampling method: we use transformation function $g_\phi(\boldsymbol{\epsilon},\mathbf{x}) = \boldsymbol{\mu} + \boldsymbol{\sigma}^2\odot \boldsymbol{\epsilon}$, and sampling function $p(\epsilon) = N(0,\textbf{I})$. Note that throughout this work we consistently only use one sample $\epsilon$ for each latent representation $\mathbf{z}$, and we therefore do not use an index for this.\\
The decoder $p(\mathbf{x}|\mathbf{z})$ is also a neural network with as input (a sample from) latent representation $\mathbf{z}$ and as output the probabilities of a Multinomial distribution, with a ReLu activation function used in each hidden layer. With one hidden layer, this would specified by:

\begin{align}
\mathbf{h_{d1}} = \text{ReLu}(\mathbf{zW_{d1}+b_{d1}})
\\
p(\mathbf{x}|\mathbf{z}) = \text{softmax} (\mathbf{h_{d1}W_{d2}}+\mathbf{b_{d2}})
\end{align}
Where the $\text{softmax}(\mathbf{x}) = \dfrac{e^{\mathbf{x}}}{\sum_{k=1}^{K}e^{x_k}}$


Discuss implications of plate difference? Discuss alternative approach with separate latent variables and/or noise for each word?\\ \\

\subsection{Objective Function}
The general form of the SGVB estimator is:

\begin{align}
\tilde{\mathcal{L}}(\boldsymbol{\theta}, \boldsymbol{\phi}, \mathbf{x_i}) = -D_{KL}(q_\phi (\mathbf{z}|\mathbf{x}_i)||p(_\theta(\mathbf{z}))  + \frac{1}{L}\sum_{l=1}^{L}\log p_\theta(\mathbf{x}_i|\mathbf{z}_i)
\end{align}

And because we consistently only use one sample from $p(\boldsymbol{\epsilon})$ per data point, this simplifies to:

\begin{align}\label{lb_summary}
\tilde{\mathcal{L}}(\boldsymbol{\theta}, \boldsymbol{\phi}, \mathbf{x_i}) = -D_{KL}(q_\phi (\mathbf{z}|\mathbf{x}_i)||p(_\theta(\mathbf{z}))  + \log p_\theta(\mathbf{x}_i|\mathbf{z}_i)
\end{align}

Because we use Gaussian latent variables with diagonal covariance, we can integrate the KL Divergence analytically as done in Kingma and Welling \cite{kingma2013auto}. \textit{Might need to elaborate on this, if not already done so in the background chapter}. Adding the expression for the Multinomial likelihood $p_\theta(\mathbf{x}_i|\mathbf{z}_i)$, we then have
\begin{align}\label{LBest}
\tilde{\mathcal{L}}(\boldsymbol{\theta}, \boldsymbol{\phi}, \mathbf{x_i}) = - \frac{1}{2}\sum\limits_{j=1}^{J}\{1+\log \sigma_{\phi ,ij}^2 - \mu_{\phi,ij}^2 - \sigma_{\phi ,ij}^2\}  + 
\sum_{k=1}^K x_{ik} \log (y_{ik})
\end{align}
\\
Notably, this is the lower bound per \textit{document}. In (bag-of-words) topic modeling, likelihood measures such as perplexity are usually per-word measures. To obtain a per-word lower bound, we must divide the total lower bound for a set of evaluated documents by the number of words in that set: 
\begin{align}\label{perwordLBest}
\tilde{\mathcal{L}_w}(\boldsymbol{\theta}, \boldsymbol{\phi}, \mathbf{X}) = \frac{1}{\sum\limits_{i=1}^{N}\sum\limits_{k=1}^{K}\mathbf{X}_{ik}}\sum\limits_{i=1}^N \tilde{\mathcal{L}}(\boldsymbol{\theta}, \boldsymbol{\phi}, \mathbf{x_i})
\end{align}

We cannot compare this lower bound estimate in \ref{perwordLBest} nor the one in \ref{LBest} directly to different models or measures and the lower bound estimate is therefore merely for model optimization. Although \ref{LBest} and \ref{perwordLBest} are functionally equivalent, a per-word lower bound is independent of average document size and more relatable to other per-word measures in topic modeling, so we prefer to use this measure when reporting experimental results over e.g. the lower bound per document.



\section{Stick Breaking Topic VAE}
Where other popular topic models such as LDA and Deep Exponential Families \textit{(refs)} have binary latent variables, the VAE approach described in section \ref{TopicVAE} has continuous latent variables. These are, as typically done, assumed to be independent and normally distributed. Using Beta distributed latent variables in a SGVB approach is not possible because, as explained in \ref{sbvae_section}, but we can use a stick-breaking VAE for topic modeling in a similar manner to the topic VAE in \ref{TopicVAE}. This approach seems particularly relevant for a Topic VAE \textit{explain why: sparsity of data? discriminative? Or just because LDA DEF?}. \\
As in section \ref{TopicVAE}, we will first describe the model specifics and then explain the resulting objective function.
\subsection{Model Specifics}
Once again, for a fully specified model, we need to define:
\begin{enumerate}
	\item The representation of our documents $\mathbf{X}$ remains unchanged.
	\item The encoder $q(\mathbf{z}|\mathbf{x})$ now encodes the parameters $\{\mathbf{a}, \mathbf{b}\}$ of a Kumaraswami distributions instead of $\boldsymbol{\mu}$ and $\boldsymbol{\sigma^2}$ of univariate Gaussians (see \ref{vae_encoding_mu} and \ref{vae_encoding_sig})
	\item $p(\mathbf{z})$, a prior  over the latent variables. This prior is now generated with the stick-breaking process described in \ref{sb_process}. Note that in order to do this, we much choose a Dirichlet prior $\alpha_0$. \textit{explain why and how this is a Dirichlet prior, see 2.2 of sbvae paper}
	\item A differentiable way of sampling $\mathbf{z}$ from $p(\mathbf{z}|\mathbf{x})$. As detailed in \ref{kum}, we can do this by sampling from the inverse CDF of the Kumaraswami distribution: $\mathbf{x} = (1-\boldsymbol{\epsilon}^{\frac{1}{\mathbf{b}}})^{\frac{1}{\mathbf{a}}}$, where $\boldsymbol{\epsilon} \sim \text{Uniform}(0,1)$
	\item The decoder $p(\mathbf{x}|\mathbf{z})$. This remains unchanged te decoder described in section \ref{TopicVAE}.
\end{enumerate}

\subsection{Objective Function}
As $ p_\theta(\mathbf{x_i}|\mathbf{z_i})$ (see equation \ref{lb_summary}) is once again modeled as a Multinomial, that part  remains unchanged. The KL Divergence $D_{KL}(q_\phi (\mathbf{z}|\mathbf{x}_i)||p(_\theta(\mathbf{z}))$ is now the KL divergence between the Dirichlet stick-breaking prior distribution and the Kumaraswami posterior distribution:

\begin{align}
KL(q(\mathbf{\mathbf{\pi}_i}|\mathbf{x}_i)||p(\mathbf{\pi}_i;\alpha_0)) = \sum_{k=1}^{K}\mathbb{E}_q [\log q(v_{i,k})] - \mathbb{E}_q \log p(v_{i,k})]
\end{align}

Truncating the infinite-dimensional distribution (stick-breaking process) by setting $\pi_i,K$ to s.t. $\sum_{k=1}^{K}\pi_{i,k} = 1$, the KL divergence can be written as:

\begin{align*}
\sum_{k=1}^{K}\mathbb{E}_q [\log q(v_{i,k})] - \mathbb{E}_q \log p(v_{i,k})] = \frac{a_0 -\alpha}{a_0}(-e-\Psi(b_\phi-\frac{1}{b_\phi}))+ \log(a_\phi b_\phi) + \log B(\alpha, \beta) \\
- \frac{b_\phi -1}{b_\phi} 
+ (\beta-1)b_\phi\sum_{m=1}^{\infty}\frac{1}{m+a_\phi b_\phi}B(\frac{m}{a_\phi},b_\phi)
\end{align*}

(is the Digamma function differentiable? If so, how to implement? Currently 2nd order taylor which works fine)\\

where $e$ is Euler’s constant, $\Psi(\cdot)$ is the Digamma function, and $B(\cdot)$ is the Beta function.
The infinite sum, which originates from a Taylor appriximation, can be approximated well by the first 2 or 3 terms. When $\beta$ is one, as when using a weighted sum of Beta distributed latent variables for a Dirichlet distribution, this last term vanishes. For a full derivation see Nalisnyck and Smith, SB-VAE.\\


\section{Dealing with large vocabulary sizes}
One problem of using a VAE approach to topic modeling is the large dimensionality of large corpora. So far, we handle this by discarding the infrequent words in the dataset. \textit{where do we talk about this? refer, or talk about it here!}  In this section we discuss approaches to include this discarded information in our VAE approach in a feasible, scalable manner.\\
The main idea is to project this very sparse data linearly onto a smaller subspace, retaining as much information as possible. Suppose we select from dataset $\mathbf{X}$ a subset of frequent words $\mathbf{X}_{fr}$ to use in the VAE approach described in section \ref{TopicVAE}. Instead of discarding the infrequent words $\mathbf{X}^{if}$, we use a lower dimensional representation of $\mathbf{X}^{if}$ concatenated with $\mathbf{X}^{fr}$. This way, the first layer of the encoder (equation \ref{he1} in chapter \ref{TopicVAE}) becomes:
\begin{align}
\mathbf{h_{e1}} = \text{ReLu}(\left(\begin{matrix}
\mathbf{\hat{x}} &
\mathbf{x}^{ld}
\end{matrix}\right)\mathbf{W_{e1}} + \mathbf{b}) \label{he1_RP}
\end{align}
Where $\mathbf{x}^{ld}$ is the lower dimensional representation of the sparse words in a document.
It is important to note that we do not attempt to model projected data $\mathbf{X}^{ld}$ explicitly in $p(x|z)$  as its probability distribution has now changed.  It therefore merely provides the encoder with additional information. In this work, we investigate two ways of achieving such a lower dimensional representation. 


\textit{explain why: computationally costly mainly in reconstruction because dense and instable in $\log p(x|z)$}.
\textit{(do we want an info-graphic on this? Or maybe just state how many words we leave out in de datasets section?)}

\subsection{Random Projection}\label{RP}
One effective way of achieving such a lower dimensional representation of sparse data is by means of a Random Projection, a method often useful for dimensionality in the context of text (see e.g. \cite{bingham2001random} for a comparison to other dimensionality reduction methods). The method is based on the Johnson-Lindenstrauss lemma \cite{frankl1988johnson}, which states that a linear mapping of a large set of high-dimensional vectors to a much lower dimensionality can approximately preserve distances between vectors. A linear mapping of $N$ sparse vectors of dimension $D$ onto a $K$-dimensional space is done by multiplication of the projection matrix $\mathbf{R}_{D\text{x}K}$:

\begin{align}
\mathbf{X}^{ld} = \mathbf{X}_{N\text{x}K} = \mathbf{R}_{D\text{x}K}\mathbf{X}_{N\text{x}D}
\end{align}

 For the Johnson-Lendenstrauss lemma to hold, the matrix $R$ must be orthogonal. However, a results by Hecht-Nielsen \cite{hecht1994context} states that vectors in high-dimensional spaces are very likely to be close to orthogonal. Often Gaussian distributed random matrices are used, and Bingham  and Mannila \cite{bingham2001random} show experimentally that a high-dimensional random matrix with Gaussian distributed values is close to orthogonal. \\
In this work we shall therefore restrict ourselves to a matrix with entries drawn from $N(0,\sigma^2)$. For efficient optimization, it is best that the entries in $\mathbf{X}^{ld}$ are in approximately the same range as $H_{e1}$ (see equation \ref{he1}). Therefore we choose $\sigma$ = $\frac{0.5}{\sqrt{\bar{n}}}$, where $\bar{n}$ is the average document length in the dataset. 

Data $\mathbf{X}_{N\text{x}D}$ is a sparse matrix, to be projected onto $K$-dimensional space through $\mathbf{X}_{N\text{x}K} = \mathbf{R}_{D\text{x}K}\mathbf{X}_{N\text{x}D}$

The $K$ columns of $\mathbf{R}_{D\text{x}K}$ are generally of unit length, and it helps if the are orthonormal, which for our application is of negligible computational cost to do. If we normalize each document s.t. the inputs represent probabilities, we can initialize the weights $R$ and $W_1$ with $N(0,1)$, s.t. each hidden unit will receive input distributed as $N(0,1)$. 

\subsection{Singular Value Decomposition}\label{SVD}
The second method of obtaining a lower dimensional representation of infrequent words $\mathbf{X}^{if}$ is by Singular Value Decomposition:

\begin{align}
\mathbf{X}^{if} = \mathbf{U\Sigma V}
\end{align}

We then use $K$ left-singular vectors with the highest singular values as $\mathbf{X}^{ld}$. We then once again use each row of $\mathbf{X}^{ld}$ as in equation \ref{he1_RP}.
\textit{perhaps a brief discussion of advatnages/disadvantages of RP compared to SVD?}
\section{Graph Convolutions for Topic VAE}

Bag-of-words data can be seen as a bipartite graph between document nodes and word nodes (add image of this graph?). From this perspective, it makes sense to look into incorporating the work by Kipf \& Welling on Graph Convolutional Networks in our approach. \\
As detailed in \ref{GCN_section}, the first layer of a GCN is given by:
\begin{align}\label{GCE_layer}
H = \text{ReLU}(\bar{D}^{-\frac{1}{2}}\bar{A}\bar{D}^{-\frac{1}{2}}FW)
\end{align}
\\
Where ReLU can be an other nonlinearity of choice. $\bar{A}$ is the (symmetric) adjacency matrix $A$ with added self connections $I$:$\bar{A} = A+I$, and $\bar{D_{ii}}=\sum_j\bar{A}_{ij}$. Further, $W$ is a trainable weight matrix similar to those in our earlier approaches. In our application, $\bar{A}$ would be of dimension $(N_d + V) \text{ x } (N_d + V)$, with both the upper left corner and the bottom right corner an identity matrix (note: explain what $A_{ij}$ is). We can rewrite \ref{GCE_layer} as:

\begin{align}
H = \text{ReLU}(X'W)
\end{align}

Where $X' = \bar{D}^{-\frac{1}{2}}\bar{A}\bar{D}^{-\frac{1}{2}}F$. A straightforward way of using Graph Convolutions is to incorporate one (or more) Graph Convolutions in our encoder as in \ref{GCE_layer}. \\ In this case, $F$ would logically reduce to $I$ since we have no node features. \\
Let us rewrite $X'W$, leaving out the self-connections $\mathbf{I}$ in $\mathbf{A}$: 
\begin{align}
X' = \bar{D}^{-\frac{1}{2}}
\left( \begin{matrix} 
0 && G \\
G^T && 0
\end{matrix} \right) \bar{D}^{-\frac{1}{2}}W = \\
\left(\begin{matrix}
\bar{G}W_1 \\
\bar{G^T}W_2
\end{matrix}\right)\label{highlow}  
\end{align}
Where $\bar{G}_{ij} = 
\frac{G_{ij}}{\sqrt{\sum_i G_{ij} \sum_{j} G_{ij}}}$ \textit{indexing isnt correct yet I think}.

We now have multiplications with two weight matrices $W_1$ and $W_2$, one that has parameters for each $N_d$ documents and one that has parameters for each word in $V$. Having parameters for each document is not scalable to large datasets, especially since a batched approach to $\bar{G^T}W_2$ is impossible.\\
One way of encoding some document-level information is to use the covariance matrix $\bar{G^T}\bar{G}$ in stead of $\bar{G^T}$. Now, the number of parameters in our first layer would scale with $(2\cdot V)$ instead of $(V+N)$. \ref{highlow} now becomes:
\\
\begin{align}
\left(\begin{matrix}
\bar{G}W_1 \\
\bar{G^T}\bar{G}W_c
\end{matrix}\right)
\end{align}
\\
Using a minibatch approach to $\bar{G^T}\bar{G}W_2$ requires calculating $\bar{G^T}\bar{G_{batch}}W_c$ for each batch, which is of complexity $O(V\text{ x }V \text{ x } h)$, where $h$ is the number of hidden units. Even for a large batch size, this is much more expensive than calculating the sparse multiplication $\bar{G}_{batch}W_1$, which is of $O(\text{nonzero}(\bar{G}_{batch}) \text{ x } h \text{ x } \text{batchsize})$. 
\\
Using the full covariance matrix for each minibatch allows for computing $\bar{G^T}\bar{G}$ only once. However, writing out the first layer for one row $\mathbf{g}$ of $\bar{G}$ shows us this approach does not combine information in $\mathbf{g}$ and $\bar{G}_T\bar{G}$:
\begin{align}
h_1 = \text{ReLU}(
\left(\begin{matrix}
\mathbf{g} \\
\bar{G^T}\bar{G}
\end{matrix}\right)W_1 +b_1)
\\
h_1 = 
\text{ReLU}(\left(\begin{matrix}
	\mathbf{g}W_{1a} \\
	\bar{G^T}\bar{G}W_{1b}
\end{matrix}\right) + b_1)
\end{align}
\\
Note that leaving out $\bar{G}^TW_2$ in \ref{highlow} leaves us with the original first layer of our encoder, except with a different TF-IDF-like normalization of our data. While this insight perhaps does not change our model much, it is worth some experiments, which we will detail in Chapter \ref{experiments}   \textit{Write out this normalization and forecast experiments.} 
\chapter{Experiments and Results}\label{experiments}
\section{General}
First we will describe the datasets used, discuss briefly the evaluation metrics reported, as well as describe the general optimization method used in our experiments.
	\subsection{Datasets}\label{datasets}
	For all methods, we ran experiments on the KOS and NY Times datasets, freely available at UCI\footnote{https://archive.ics.uci.edu/ml/datasets/Bag+of+Words}. Both datasets contain only words that occur more than ten times in the whole dataset. The KOS dataset contains 3430 documents and has a vocabulary size of 6906. The dataset was split into 3300 training documents and 130 test documents. The NY times dataset consists of 300,000 documents and has a vocabulary size of 102,660 words. For the NY Times dataset, we only use words that occur over 3,000 times in the dataset, which are 5319 unique words. This makes training time and model evaluation a lot faster, as both scale approximately linearly with input dimensionality. \textit{Leaving out infrequent words only has a minor effect on the perplexity of bag-of-word topic model perplexity, mainly due to Zipf's law (Cite Kobayashi).} For this dataset, a test set of 1,000 documents was used.
	\\
	\subsection{Evaluation}
	
	Evaluating Topic models is often done by calculating the perplexity of held out words on the test set (e.g. \cite{blei2003latent, newman2007distributed, ranganath2015deep}). In this work, held-out perplexity is calculated for the test set as in Newman \& Welling \cite{newman2007distributed} by using half the words, randomly sampled, in each document for inference (i.e. calculating $p(\mathbf{z}|\mathbf{x})$ ). Let $\mathbf{x}_{i}^{(s)}$ be the half of document $\mathbf{x}_i$ used for inference, and $\mathbf{x}_{i}^{(u)}$ be the other half of the document, used for evaluation. The average per-word perplexity of the unseen words $\mathbf{X^{(u)}}$ in the test documents $X$ under the word-probabilities $p(X|Z)$, where $Z^s \sim p(\mathbf{Z}|X^{(s)})$ is then given by:
	
	
	\begin{align}
	\text{perplexity} =  \frac{1}{\sum\limits_{i=1}^{N}\sum\limits_{k=1}^{K}x_{ik}}\sum\limits_{i=1}^N\sum\limits_{k=1}^{K} \log p(x_{k}|\mathbf{z}_{i}^{s})x_{ik}^{u}
	\end{align}\\
	\textit{Note: take another look at perplexity calculation in DEF}
	\\
	Add some stuff on reporting the lower bound. Refer to earlier elaboration on LB per word.
	
	\subsection{General Optimization Method}\label{optim_section}
	We use Adam, learning rate 0.003 unless otherwise stated. Batch size 50 for KOS and 200 for NY Times. Describe weights initialization. KL Divergence is linearly increased during first 100 epochs of training. (different for NY Times?).
	

	\section{Experiments}
	\subsection{Topic VAE}
	
	From Hinton: 24.12 \textit{A recipe for choosing the number of hidden units
		Assuming that the main issue is overfitting rather than the amount of computation at training or
		test time, estimate how many bits it would take to describe each data-vector if you were using a good
		model (i.e. estimate the typical negative log2 probability of a data vector under a good model). Then
		multiply that estimate by the number of training cases and use a number of parameters that is about
		an order of magnitude smaller. If you are using a sparsity target that is very small, you may be able
		to use more hidden units. If the training cases are highly redundant, as they typically will be for very
		big training sets, you need to use fewer parameters.}\\

	In this section we outline our experiments with the VAE Topic Model. \textit{(brief summary of section?)} 
	
	
	
	\subsubsection{Experiment 1: A First Model}
	Describe choice of number of latent variables, hidden units to start with (200e, 20d). Report lower bound train and test over time, as well as held-out test perplexity. For KOS and for NY Times
	
	\subsubsection{Hyperparameter Optimization}\label{HO_section}
	\textit{How to motivate parameter chocies for which we do not have saved experimental results, such as gradual KLD increase, initialization and learning rates, reLU's?)}
	For each dataset, we train models with different encoder and decoder structures in order to optimize these. We use 8 and 32 latent variables for the KOS and NY Times datasets, respectively. We use the whole KOS dataset as detailed in \ref{datasets}, but only use 100,000 NY Times documents for training. Many other hyperparameters such as optimization method and details (see \ref{optim_section}), initialization of parameters, and gradual increase of KL Divergence, are chosen based on preliminary experiments. These are all parameters that might influence convergence, but which theoretically do not influence the performance of a converged model. In case of nonlinearity types and choice of prior $p(\mathbf{z}$), we do not vary these because we did not note a significant influence of these choices on model performance in preliminary experiments.\\
	We report all trained models and their performance in section \ref{ho}.
	
	
	\subsubsection{Varying the Training Set Size}
	One question regarding a neural SGVB approach like ours is how well this is suited for different amounts of available training data. Our approach is assumed to be applicable to, and effective for,  large amounts of training data. Therefore we want to know if a model is better, in general, when trained on more data. Another question worth answering is how much training time scales with the size of the dataset. And lastly, it is useful to gain some insight in how prone a model is to overfitting with a relatively small dataset.\\
	To answer these questions, we train a model with a relatively high capacity that was shown to perform well in the results of the hyperparameter optimization experiments in section \ref{HO_section}. We use training set sizes between 5k and 300k documents and test the models on the same test set of 1k documents. We compare effectiveness of converged models by reporting the test lowerbound and 50\% held-out test set perplexity. To get a feel for overfitting effects, we also report the difference between the best test lowerbound and perplexity during training to the values after convergence. Due to the large difference in training set size, a different number of epochs is used for each model. We also show the total number of documents evaluated before convergence (or overfitting) as a function of training set size to answer how training time scales with training set size.
		
	\subsection{Stickbreaking Topic VAE}\label{sbvae_exp}
	
	We also run experiments with stick-breaking VAE's as detailed in Section \ref{sbvae_section}. For both the KOS and NY datasets we train a model with an architecture that performed well with Gaussian latent variables (NY: 400200e100d, KOS: 200100e0d) in section \ref{HO_section} on the same training set used in that section for values of concentration parameter $a= \{1,3,6\}$ (see relevant equation). We show these results and a comparison to other methods in Section \ref{comp_methods}.
	
	\subsection{Large Vocabulary Sizes}
	For testing the benefit of using the infrequent words in the encoder with either a random projection (see \ref{RP}) and SVD  we use a similar approach. We use the same model architectures as in \ref{sbvae_exp}. All all words not used so far in experiments on the NY Times dataset (as explained in \ref{datasets}), are represented by either a random projection of these words or the largest singular values, as described in \ref{RP} and \ref{SVD}, respectively. For the NY Times dataset we use dimensionality $K = 200$. \\
	\textit{do we even want to do this for KOS? Do we need to run more experiments, for example extra hidden layer for large model?)}
	
	\subsection{Word-Document Normalization}
	
	\subsection{Comparison to Other Methods}
	LDA, DEF
	
	
	
	\section{Results}
	
	\subsection{Topic VAE}
	
	\subsubsection{A First Model}
	
	\subsubsection{Hyperparameter Optimization}\label{ho}
	
	Figure \ref{HO_KOS} and \ref{HO_NY} detail the lower bound of both the train and test sets of the used datasets for converged models with different encoder/decoder structures. We also report the the 50\% held-out perplexity on the test set.\\
	Note that the encoder typically has many more parameters than the decoder. This if because the encoder is not prone to overfitting. \textit{explain} 
	
	\begin{figure}
	\begin{tabular}{l l|l l |l l }\label{HO_NY} 		
		Encoder & Decoder & LB Train & LB Test & Perplexity  	\\
		\hline
		400 & -		& -7.354 & -7.378941743 & 1492 \\

		400 & 50	& -7.325 & -7.358300133 & 1478 \\

		1000 & -	& -7.341 & -7.379 & 1513 \\
	
		400-200 & -	& -7.345 & -7.375 & 1493 \\

		400-200 & 100	& -7.284 & -7.327 & 1429 \\
		
		600-300 & 300	&  &  &  \\

		1000-600-300 & 200	& -7.212 & -7.308 & 1415 \\
		
		1000-600-300 & 500	& -7.180 & \color{red}-7.297 & \color{red}1384 \\
		
		1000-600-300 & 200-500	& \color{red} -7.161 & -7.324 & 1448 \\

	\end{tabular}
\end{figure}	
	\subsection{Dataset Size}
	As described in \ref{size_section}, we ran experiments on increasing training set sizes of the NY Times dataset, using the same hyperparameters each time. The results are shown in Figure ...

	\subsection{Additions to Topic VAE}\label{comp_methods}
	Stick Breaking, Random Projections, Batch Normalization, different normalization.
	\begin{figure}
		\begin{tabular}{l l l l | l l}\label{comp_ny} 		
			Architecture* & Latent Variables & Infrequent words & Batch Normalization & Test lower bound & Perplexity  	\\ \hline
				&	Gaussian	&	Unused				&	no	&	-7.308 	& 1415 \\ 
				&	Gaussian	&	Unused				&	yes  &	 	&  \\ 
			A	&	Gaussian	&	Random Projection	&	no	&	 	&   \\ 
				&	Gaussian 	&	SVD					& no	&	 	& \\ 
				&	Stick-Breaking	&	Unused	&	yes &	& \\  \hline
&	Gaussian	&	Unused				&	no	&	 	&  \\ 
&	Gaussian	&	Unused				&	yes  &	 	&  \\ 
B	&	Gaussian	&	Random Projection	&	no	&	 	&   \\ 
&	Gaussian 	&	SVD					& no	&	 	& \\ 
&	Stick-Breaking	&	Unused	&	yes &	& \\  \hline
	
		\end{tabular}
	\end{figure}
	

	

		
\chapter{Discussion}
What are advantages/disadvantages of our method(s) compared to existing approaches?\\
How does our method compare to other methods as far as perplexity goes? Can we understand this?\\
For which purposes would we recommend (one of) our approaches, and why? \\
Thoughts: multiple layers of latent variables (or even a different architecture?) possible.
\section{Conclusion}
\section{Future Work}
\bibliography{ref}
\end{document}