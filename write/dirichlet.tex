\documentclass{article}


\usepackage{amsmath,amssymb}
%\usepackage{dsfont} %install texlive-fonts-extra 
\usepackage{tikz}
\usetikzlibrary{bayesnet}

\author{Otto Fabius}
\title{Dirichlet Latent Variables}
\begin{document}

\maketitle

\section{KL Divergence}

We would like to use Dirichlet latent variables, with a Dirichlet prior. This means we need (1) the KLD between the prior and the approximate posterior: $KL(q(\mathbf{z}|\mathbf{x}^{(n)}||p(\mathbf{z})$ and a method to take samples from $q(\mathbf{z}$. The KLD can be calculated analytically.
With $p(\mathbf{z}) = Dir(\alpha)$ and $q(\mathbf{z}|\mathbf{x}^{(n)} = Dir(\beta)$:
\begin{align}
KL(q(\mathbf{z}|\mathbf{x}^{(n)}) = \log \Gamma(\beta_0) - \sum_{k=1}^K\log\Gamma(\beta_k)-\log\Gamma(\alpha_0)+ \\
\sum_{k=1}^K\log\Gamma(\alpha_k)+ \sum_{k=1}^K(\beta_k - \alpha_k)(\psi(\beta_k)-\psi(\beta_0))
\end{align}

Where

\begin{align}
\alpha_0 = \sum_{k=1}^{K}\alpha_k \text{ and } \beta_0 = \sum_{k=1}^{K}\beta_k
\end{align}

Regarding the choice of $\alpha$, a naive choice seems to be a symmetric prior with $\alpha < 1$, similar to LDA (e.g. $0.3$). With this approach, we can even optimize $\alpha$ (whether we want to have a symmetric prior or not), although optimizing an asymmetric prior might be prone to overfitting.

\section{Reparametrization}

In order to obtain gradients of our latent variables, we write our Dirichlet distribution $z \sim Dir(\alpha)$ as a weighted sum of Gamma variates: $z_k = \frac{y_k}{\sum_{k=1}^{K}}$ with $y_k \sim \text{Gamma}(\alpha_k,1)$. \\
The sum of $n$  exponential $(\beta)$ random variables is a $Gamma(n, \beta)$ random variable. Write each Gamma dist variable as sum of exp dist variables with $\lambda = 1$. So $y_k \sim Gamma(\alpha_k,1)$ can be done as $y_k = \sum_{n=1}^{N}v_n$ with $v_n \sim e^{-x}$. Can we find a continuous function (or approximation) $f(n,x)$ s.t. we can sample $y_k$ from $f(n,x)$ directly? NB we must still be able to apply the reparametrization trick to it, so it must be location-scale family or have a tractable inverse CDF (quantile)! Alternative idea: train a (small, differentiable) neural network to approximate the quantile function of the Gamma distribution! (input: different values of $\alpha_k$  and $p$, i.e. values of $\epsilon$. Obtain these simply by calculating the values with CDF function).


\end{document}