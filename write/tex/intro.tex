\chapter{Introduction}

Generative modeling , neural nets in a few lines.

%A class of graphical models is that of Generative models, where the data in question is assumed to be generated by some process involving underlying factors (latent variables). 

 

\section{Variational Bayes}

Explain VB

\subsection{SGVB}
 
- General purpose introduction of sgvb . \\

- areas of success of sgvb.

\section{Bag-of-words Topic Modelling}
some paragraphs on (generative) topic modelling. 
Not too much detail but explain key idea and perhaps mention some pros and cons of current methods. 
\section{Research question}
Explain that we want to use sgvb for topic modelling. Introduce research question:
Can we perform large-scale, efficient, high-quality inference on bag-of-words representations of documents with sgvb? \\
Briefly discuss the advantages of this approach compared to other methods in topic modelling (one paragraph)
-	How do we deal with large vocabulary size?\\
-	What consequences do sparsity have/how can they be overcome? \\
-	What is learned in the (continuous) latent representation?


